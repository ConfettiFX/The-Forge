\documentclass{article}
\usepackage{graphicx}
\usepackage{amsmath}
\usepackage{listings}
\usepackage{color}
\usepackage{float}
\usepackage{geometry}
\usepackage{hyperref}

 \geometry{
 a4paper,
 total={170mm,257mm},
 left=20mm,
 top=20mm,
textwidth = 170mm
 }

\floatstyle{boxed} 
\restylefloat{figure}

\lstset{
 breaklines        = true,
 breakatwhitespace = true,
 breakindent       = 2ex,
 escapechar        = @,
 numbers           = none,
 language		   = C++,
}

%tools for typograpgy convention
\renewcommand{\Vec}[1]{#1}
\newcommand{\Mat}[1]{#1}
\newcommand{\MatSubText}[2]{#1_\text{#2}}
\newcommand{\BuildMat}[1]{\left[\begin{matrix}#1\end{matrix}\right]}
\newcommand{\transpose}{^{\mathstrut\scriptscriptstyle{\top}}}
\let\lstInline\lstinline %fixed syntax highlite in texmaker
\newcommand{\InlCode}[1]{\lstInline[basicstyle=\ttfamily]{#1}}


\hypersetup{
    colorlinks=true,
    linkcolor=blue,
    filecolor=magenta,      
    urlcolor=cyan,
}

%code snippet colors
\definecolor{dkgreen}{rgb}{0,0.6,0}
\definecolor{gray}{rgb}{0.5,0.5,0.5}
\definecolor{mauve}{rgb}{0.58,0,0.82}
\lstset{frame=tb,
  language=C++,
  aboveskip=3mm,
  belowskip=3mm,
  showstringspaces=false,
  columns=flexible,
  basicstyle={\small\ttfamily},
  numbers=none,
  numberstyle=\tiny\color{gray},
  keywordstyle=\color{blue},
  commentstyle=\color{dkgreen},
  stringstyle=\color{mauve},
  breaklines=true,
  breakatwhitespace=true,
  tabsize=3
}

\begin{document}
\title{Conffetti Math Conventions}
\author{Marijn Tamis, Volkan Ilbeyli}
\maketitle

\begin{abstract}
This document will describe the math conventions used in Confetti codebase - The Forge.
\end{abstract}

% 1. MATRICES & VECTORS
%-------------------------------------------------------------------------------------------------------------------------------------
\section{Matrices and Vectors}
\subsection{Matrix multiplication}
All matrices are defined and used as column major matrices.
This means that vectors (specifically those representing positions and directions) are column vectors.
Transforming vector $\Vec{v}$ with transformation matrix $\Mat{M}$ is written as follows:
\begin{align*}
\Vec{v}' &= \Mat{M} \Vec{v}\\
\BuildMat{
m_{11}x+m_{12}y+m_{13}z\\
m_{21}x+m_{22}y+m_{23}z\\
m_{31}x+m_{32}y+m_{33}z
}
&=
\BuildMat{
m_{11}&m_{12}&m_{13}\\
m_{21}&m_{22}&m_{23}\\
m_{31}&m_{32}&m_{33}
}
\BuildMat{x\\y\\z}
\end{align*}
It follows from this convention and the associative property of matrices that the sequential transformations can be combined in a single matrix as follows:
\begin{align*}
\left(\Mat{M}_2\Mat{M}_1\right)\Vec{v} &= \Mat{M}_2\left(\Mat{M}_1\Vec{v}\right)
\end{align*}
Where $\Vec{v}$ is first transformed by $\Mat{M}_1$ and then by $\Mat{M}_2$.

\subsection{Matrix memory layout}
The elements in the matrices are laid out in memory in column major order.
Note that this is an implementation detail which is unrelated to the notational convention.\footnote{Although it is easier to utilize vector instructions when the conventions match.}
Figure \ref{tabular:mat4MemoryLayout} shows how a $4\times 4$ matrix is laid out in continuous memory.\\
\begin{figure}[b]
\centering
\begin{tabular}{cc}
Matrix&Memory Layout\\
$
\BuildMat{
m_{11}&m_{12}&m_{13}&m_{14}\\
m_{21}&m_{22}&m_{23}&m_{24}\\
m_{31}&m_{32}&m_{33}&m_{34}\\
m_{41}&m_{42}&m_{43}&m_{44}\\
}
$
&
\begin{tabular}{|c|c|c|c|c|}
\hline
&\texttt{0x0}&\texttt{0x4}&\texttt{0x8}&\texttt{0xC}\\
\hline
\texttt{0x00}&$m_{11}$&$m_{21}$&$m_{31}$&$m_{41}$\\
\hline
\texttt{0x10}&$m_{12}$&$m_{22}$&$m_{32}$&$m_{42}$\\
\hline
\texttt{0x20}&$m_{13}$&$m_{23}$&$m_{33}$&$m_{43}$\\
\hline
\texttt{0x30}&$m_{14}$&$m_{24}$&$m_{34}$&$m_{44}$\\
\hline
\end{tabular}
\\
\end{tabular}
\caption{The memory layout of \InlCode{mat4}}
\label{tabular:mat4MemoryLayout}
\end{figure}
\\Also note that the constructor of \InlCode{mat4}\footnote{\InlCode{mat4} is a \InlCode{typedef} for the Sony Matrix4 class} takes 4 \InlCode{vec4}'s which are the columns of the matrix.
This makes it look like the matrix is written transposed in the code.

\subsection{Vector Types}
The math library used in The Forge codebase provides two types of vectors: \InlCode{vec2/3/4} and \InlCode{float2/3/4}.
\begin{itemize}
	\item \InlCode{vec2/3/4} is used for optimized math calculations and may contain padding. A \InlCode{vec3} has a \InlCode{w} component for alignment and optimization reasons, \InlCode{sizeof(vec3)} will return 16 Bytes as the size.
	\item \InlCode{float2/3/4} guarantees struct sizes and is used for storing data. A \InlCode{float3} will contain 3 floating-point elements and will be 12 Bytes in size.
\end{itemize}

% 2. COORDINATE SYSTEM
%-------------------------------------------------------------------------------------------------------------------------------------
\section{Coordinate System}
\subsection{World Space}
The coordinate system used for world space coordinates is left handed when the \InlCode{mat4::perspective()} and \InlCode{mat4::orthographic()} are used.
Functions like \InlCode{mat4::rotationX()} will generate a matrix that rotates vectors in clockwise direction when the axis points towards the observer (as described by the left hand rule).
The up vector is $\Vec{\hat y} = \BuildMat{0&1&0}\transpose$.
$\Vec{\hat z} = \BuildMat{0&0&1}\transpose$ points away from the camera if $\Vec{\hat x} = \BuildMat{1&0&0}\transpose$ points towards the right.

\subsection{Clip Space}
The clip space is also left handed with $\Vec{\hat y}$ up, $\Vec{\hat x}$ right, and $\Vec{\hat z}$ away from the camera.
The visible volume is defined by the following ranges:
\begin{align*}
-w<x<w\\
-w<y<w\\
0\leq z<w
\end{align*}
After clipping the perspective divide is done to arrive at normalized device coordinates. We currently use the Direct3D clip space when outputting vertex coordinates in the shader. Notice that after projection, the DirectX projection model maps the Z component into [0, 1] range, unlike the OpenGL projection model which maps the Z component into [-1, 1] range. 

\subsection{Screen space}
Screen space coordinates (as passed to \InlCode{BaseApp::onMouseMove()}) are left handed, $\Vec{\hat x}$ points towards the right and $\Vec{\hat y}$ points down. The origin (0, 0) is the upper left corner, (WindowWidth, WindowHeight) is the bottom right corner.


\subsection{Examples}
\subsubsection{Model View Projection Matrix}
The model view projection matrix $\MatSubText{M}{mvp}$ is defined as follows:
\begin{equation*}
\MatSubText{M}{mvp} = \MatSubText{M}{Projection}\MatSubText{M}{View}\MatSubText{M}{Model}
\end{equation*}

\begin{lstlisting}	
// Example Application Code
const float aspectInverse = (float)gWindowHeight / (float)gWindowWidth;
const float fovh = gPi / 2.0f;

mat4 projMat = mat4::perspective(fovh, aspectInverse, 0.1f, 1000.0f);
mat4 viewMat = pCameraController->getViewMatrix();

gUniformData.mProjectView = projMat * viewMat;
\end{lstlisting}

\begin{lstlisting}	
// Example Shader Code
cbuffer uniformBlock : register(b0)
{
	matrix viewProj;
	matrix world[MAX_INSTANCES];
}

VSOutput VSMain(VSInput input, uint InstanceID : SV_InstanceID)
{
	VSOutput result;

	// Transform matrices w/ World(Model)-View-Projection matrix
	matrix mWVP = mul(viewProj, world[InstanceID]);
	result.Position = mul(mWVP, input.Position);
}
\end{lstlisting}

\subsubsection{View Matrix}
The view matrix for the default example camera is constructed as follows:
\begin{equation*}
\MatSubText{M}{View} = \MatSubText{M}{CamRotation}^{-1}\MatSubText{M}{CamTranslation}^{-1}
\end{equation*}
Example Code:
\begin{lstlisting}
mat4 Rotation = mat4::rotateXY(-wx,-wy);
mat4 Translation = mat4::translation(-camPos);
mat4 View = Rotation * Translation;
\end{lstlisting}
Note that we did not invert any matrix. The inverse of \InlCode{rotateYX(wy,wx)} is \InlCode{rotateXY(-wx,-wy)}, and the inverse of \InlCode{mat4::translation(camPos)} is \InlCode{mat4::translation(-camPos)}.


% 3. EXTENDING THE MATH LIBRARY
%-------------------------------------------------------------------------------------------------------------------------------------
\section{Using \& Extending the Math Library}
\subsection{Project Structure}

The Forge uses \href{https://github.com/glampert/vectormath}{Modified Sony Math} library, an open source math library from GitHub. This is a cleaned up version of the Sony Math Library, removing some legacy interfaces such as PS3. The library is located at \path{TheForge\Common_3\ThirdParty\OpenSource\ModifiedSonyMath\} folder and contains the math function declarations and definitions. The application code is not directly using this path. Instead, the \path{OS\Math\} handles inclusion of the math functionality while also taking care of platform-specific includes.\\\\
The structure of the library is as follows:
\begin{itemize}
	\item \path{TheForge\Common_3\ThirdParty\OpenSource\ModifiedSonyMath\}
	\begin{itemize}
	\item \path{scalar\}	- This folder contains the implementation of the library functions.
	\item \path{sse\} - Every function in the library also has an implementation using SSE intrinsics, in this folder.
	\item \path{vectormath.hpp} - Main header file that chooses the Scalar or the SSE version of the library based on the hardware support.
	\end{itemize}
	\item \path{TheForge\Common_3\OS\Math\}
	\begin{itemize}
	\item \path{MathTypes.h} - This is the header file the application code should include for math structs and functions. It includes the necessary vector/matrix headers.
	\item \path{vmInclude.h} - Chooses a math library header to include based on the application platform\footnote{This can be the \path{vectormath.hpp} file mentioned above if Desktop is the platform the application is compiled for.}. This header is used by intermediate header files (vec2/3/4.h \& mat2/3/4.h).
	\item \path{vec2/3/4.h} \& \path{mat2/3/4.h} - These headers contain the typedefs for matrix and vector types. The definitions of these types are in the \path{TheForge\Common_3\ThirdParty\OpenSource\ModifiedSonyMath\} directory.
	\end{itemize}
\end{itemize}

\subsection{Adding New Math Functionality}
In order to have a smoother experience updating the open source math library in the future, we have marked the header files where we make additions to the open source library. For example, search for \InlCode{#ConfettiMathExtensions}\footnote{Make sure to select Entire Solution ( Including External Items ) as the 'Look In' search option.} tag in the code base to see how extensions are handled.\\\\ 
To keep the codebase maintainable and consistent, one should check if the required math functionality exists in the current math library (\path{TheForge\Common_3\ThirdParty\OpenSource\ModifiedSonyMath\}) before creating a new function or file. If the math library should be extended to add the missing functinoality, the code should be added to a proper place in the \path{ModifiedSonyMath} directory. \textbf{Note that the developer should also implement the SSE\footnote{Intel's Guide for SSE Instructions: \url{https://software.intel.com/sites/landingpage/IntrinsicsGuide/\#techs=SSE}} version of the function that is being added\footnote{See the implementation of \InlCode{mat4::perspective()} as an example.}}.


\end{document}
